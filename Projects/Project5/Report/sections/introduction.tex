\section{Introduction}
Modelling disease is as relevant as ever as COVID-19 is having a signifficant impact globally.
While efforts are being made to reduce the spread and infection, the fear of the unknown, or of what we don't understand, brings out good, bad and at times very questionable protocols and policies to try to gain control over the pandemic.
Disease behaviour seems to not be very well understood by most people, and saving face, showing a willingness to act, scapegoating certain groups with no voice and keeping the industry or economy alive (to some few or the many's benefit), seems to get in the way of policies based on hard science, models, simulations and expert advice from researchers.
With this emotionally loaded motivation, largely based on general contempt for how horrible this semester has been, an escape from reality through a deep dive into disease modeling to try to scientifically understand why everything sucks becomes the perfect coping mechanism for a nerd not in touch with his feelings. 

To model disease, the SIRS model is used with the addition of vital parameters, seasonal parameters and vaccines. 
Monte Carlo simulations and a fourth order Runge-Kutta solver developed in c++, with plotting methods in python using matplotlib. 
With a theoretical background with expected values for the simplest case, the numerical methods are validated, tested and benchmarked. 
Through plots, insight is gained into how disease dvelops, with vital parameters, seasonal parameters and vaccination.
Lastly, the plots are discussed, and the two methods are compared. Lastly, the main findings are summarised in the conclusion.